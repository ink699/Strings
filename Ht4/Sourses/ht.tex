\documentclass[12pt,a4paper]{article}
\usepackage[utf8]{inputenc}
\usepackage[russian]{babel}
\usepackage[OT1]{fontenc}
\usepackage{amsmath}
\usepackage{amsfonts}
\usepackage{amssymb}
\usepackage{graphicx}
\usepackage{array}
\usepackage{cancel}
\usepackage{caption}
\usepackage{wrapfig}
\usepackage{secdot}
\usepackage{indentfirst}
\usepackage[left=1.5cm,right=1.5cm,top=0.3cm,bottom=1.5cm,includefoot,footskip=1.5cm]{geometry}
\usepackage{psfrag}
\newcommand{\tr}{\mathop{\mathrm{tr}}\nolimits}
\newcommand{\p}{\partial}
\DeclareMathOperator\arcth{arcth}
\begin{document}
\textbf{
\begin{flushright}
Илья Кочергин, 626 группа
\end{flushright}}
\paragraph{\Large Листок 4}
\section{Вопросы}
\subsection*{Вопрос 1}
\noindent Представлением Лакса, или $LA$~--- парой, называется запись некоторой системы дифференциальных уравнений в виде
\begin{equation}
\dot{L} = [L,A],
\end{equation}
где $L$ и $A$~--- матрицы, или, в более общем случае, дифференциальные операторы.
\subsection*{Вопрос 2}
\noindent Для цепочки Тоды (для простоты рассмотрен случай $N = 3$) $L$ и $A$~--- следующие матрицы:
\begin{equation}
\begin{gathered}
L = \begin{pmatrix}
p_1&x_1&x_3\\
x_1&p_2&x_2\\
x_3&x_2&p_3\\
\end{pmatrix};\\
\\
A = \begin{pmatrix}
0&x_1&-x_3\\
-x_1&0&x_2\\
x_3&-x_2&0\\
\end{pmatrix}.
\end{gathered}
\end{equation}
Для уравнения KdV ими будут дифференциальные операторы:
\begin{equation}
\begin{gathered}
L = -\partial_x^2+u\\
A = 4\partial_x^3-6u\partial_x-3u'_x
\end{gathered}
\end{equation}
\section{Упражнения}
\subsection*{Упражнение 1}
\noindent Рассмотрим матрицу $B = L^n$. Ее производная по времени имеет вид
\begin{equation}
\dot{B} = \dot{L}^n = nL^{n-1}\dot{L} = nL^{n-1}[L,A].
\end{equation}
Отсюда
\begin{equation}
\tr L^n = n\tr L^{n-1}(\tr(AL)-\tr(LA)) = n(\tr(L^{n}A) - \tr(L^{n-1}AL))=0,
\end{equation}
т.к. след не зависит от порядка перемножения. Отсюда $\tr(L^n)$ - интеграл движения при любом $n$. Т.к. след выражается через собственные значения, то число линейно-независимых интегралов равно числу собственных значений $L$ (и они также сохраняются).
\section{Задачи}
\subsection*{Задача 1}
\noindent Рассмотрим уравнение KdV:
\begin{equation}
u'''_{xxx}+6uu'_x+u'_t=0.
\end{equation} 
Произведем замену
\begin{equation}
a = x-vt,
\end{equation}
и будем искать решение вида
\begin{equation}
u(x,t) = f(a).
\end{equation} 
Тогда производные $u$ перезаписываются в следующей форме:
\begin{equation}
\begin{gathered}
u'_x = f'a'_x=f'\\
u'_t = f'a'_t=-vf'.
\end{gathered}
\end{equation}
В новых обозначениях уравнение KdV примет вид
\begin{equation}
-vf'+f'''+6ff'=0.
\end{equation}
Это уравнение легко интегрируется, получаем
\begin{equation}
-vf + f'' + 3f^2 + C_1=0.
\end{equation}
Для дальнейшего интегрирования сначала домножим уравнение на $f'$, в итоге получим
\begin{equation}
-\frac{1}{2}f^2+\frac{1}{2}f'^2+f^3+C_1f+C_2=0.
\end{equation}
При произвольных $C_1$ и $C_2$ невозможно решить это уравнение в элементарных функциях. Примем
\begin{equation}
C_1=C_2=0,
\end{equation}
тогда приходим к выражению
\begin{equation}
f' = \sqrt{vf^2-2f^3},
\end{equation}
для решения необходимо найти интеграл
\begin{equation}
\int \frac{d f}{f\sqrt{v-2f}}.
\end{equation}
Заменой
\begin{equation}
u = v-2f
\end{equation}
он приводится к виду
\begin{equation}
\int \frac{d u}{u^{\frac{3}{2}}-v\sqrt{u}}=/\tau = \sqrt{u}/ = 2\int\frac{d\tau}{\tau^2-v}=-\frac{2}{\sqrt{v}}\arcth\frac{\tau}{\sqrt{v}}.
\end{equation}
Отсюда путем обратных замен получаем решение:
\begin{equation}
-\arcth\sqrt{1-\frac{2f}{v}} =\frac{1}{2}\sqrt{v}(a + C).
\end{equation}
Из этого выражения получаем:
\begin{equation}
1-\frac{2f}{v} = \th^2\left(\frac{1}{2}\sqrt{v}(a + C)\right).
\end{equation}
Применив тригонометрические свойства и обозначив
\begin{equation}
a_0 = -C,
\end{equation}
получим окончательное решение:
\begin{equation}
f = \frac{1}{2}\frac{v}{\ch^2\left(\frac{1}{2}\sqrt{v}(a - a_0)\right)}.
\end{equation}
\subsection{Задача 2}
\noindent Вычислим $[L,A]\psi$:
\begin{equation}
\begin{gathered}
~[L,A]\psi = (\p_x^2+u)(\psi'''_{xxx}+v\psi'_x+w\psi) - (\p_x^3+v\p_x+w)(\psi''_{xx}+u\psi) = \\
=\psi_{xx}''(2v_x'-3u_x')+\psi'_x(v''_{xx}+2w'_x-3u''_{xx})+\psi(w''_{xx}-u'''_{xxx}-vu'_x).
\end{gathered}
\end{equation}
Отсюда легко получить условия, при которых $[L,A]$ не содержит операторов дифференцирования:
\begin{equation}
\begin{cases}
2v'_x-3u'_x=0\\
v''_{xx}+2w'_x-3u''_{xx}=0.\\
\end{cases}
\end{equation}
Эта система несложно разрешается относительно одного из параметров, пусть это будет $u$. Тогда получим:
\begin{equation}
\begin{cases}
v = \frac{3}{2}u\\
w = \frac{3}{4}u'.\\
\end{cases}
\end{equation}
Тогда уравнение Лакса примет вид
\begin{equation}
u'_t=-\frac{3}{2}u'''_{xxx}-\frac{3}{2}uu'_x,
\end{equation}
то есть
\begin{equation}
u'_t+\frac{3}{2}u'''_{xxx}+\frac{3}{2}uu'_x=0.
\end{equation}
\end{document}