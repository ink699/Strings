\documentclass[12pt,a4paper]{article}
\usepackage[utf8]{inputenc}
\usepackage[russian]{babel}
\usepackage[OT1]{fontenc}
\usepackage{amsmath}
\usepackage{amsfonts}
\usepackage{amssymb}
\usepackage{graphicx}
\usepackage{array}
\usepackage{cancel}
\usepackage{caption}
\usepackage{wrapfig}
\usepackage{secdot}
\usepackage{indentfirst}
\usepackage[left=1.5cm,right=1.5cm,top=0.3cm,bottom=1.5cm,includefoot,footskip=1.5cm]{geometry}
\usepackage{psfrag}
\newcommand{\tr}{\mathop{\mathrm{tr}}\nolimits}
\newcommand{\p}{\partial}
\DeclareMathOperator\arcth{arcth}
\begin{document}
\textbf{
\begin{flushright}
Илья Кочергин, 626 группа
\end{flushright}}
\paragraph{\Large Листок 5}
\section{Вопросы}
\subsection*{Вопрос 1}
Симметрией называется инвариантность действия относительно некоторого преобразования $x$ и $\phi$. При этом если симметрия является непрерывной, то есть действие инвариантно относительно некоторой непрерывной группы преобразований, то этой симметрии соответствует сохраняющийся ток (т. Нетер).  
\end{document}